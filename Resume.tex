\documentclass[a4paper]{article}
\author{Nishant Gupta}
\usepackage{array}
\usepackage[margin=2cm]{geometry}
\usepackage{longtable}
\usepackage{lmodern}
\usepackage[super]{nth}
\begin{document}
\fontfamily{phv}

%%%%%%%%%%%%%%%%%%%%%%%%%%%%%%%% Name and Address %%%%%%%%%%%%%%%%%%%%%%%%% 
\section*{\textbf\Huge Nishant Gupta}
\hrule
\vspace{2mm}
Third Year Student  \hfill {I-307/5,IIT Kanpur} \newline
Computer Science and Engineering \hfill ${nishgu@iitk.ac.in}$ \newline
IIT Kanpur \hfill  {7275726006} \\ \\
%%%%%%%%%%%%%%%%%%%%%%%%%%%%%%%%%%%%%%%%%%%%%%%%%%%%%%%%%%%%%%%%%%%%%%%%%%% 

\makeatletter
\newcommand{\thickhline}{%
    \noalign {\ifnum 0=`}\fi \hrule height 1pt
    \futurelet \reserved@a \@xhline
}

\makeatletter
\newcommand{\thinhline}{%
    \noalign {\ifnum 0=`}\fi \hrule height 0.2pt
    \futurelet \reserved@a \@xhline
}


%%%%%%%%%%%%%%%%%%%%%%%%%%%%%%%% Main Content  %%%%%%%%%%%%%%%%%%%%%%%%%%
\begin{center}
  \begin{tabular}{|c|c|c|c|} 
    \thinhline
    {\centering Year} & {\centering Degree} & {\centering Institute} & {\centering CPI} \\  
    \thinhline
    2017 (expected) & {B.Tech,  Computer Science} & IIT Kanpur  & 9.2/10.0  \\ 
    \thinhline
    2013 & {HSCE} & Shivpuri Public School,Ashoknagar & 			93.60\%	\\
    \thinhline
    2011 & AISSCE & Tara Sadan School,Ashoknagar  & 9.6/10.0   \\
    \thinhline

  \end{tabular}
\end{center}



\centering
\begin{longtable}{@{}m{3.0cm}m{14cm}@{}}
  % \begin{tabular}{@{}m{3.0cm}m{14cm}@{}}

  \textrm{\textsc {Scholastic Achievements}} & 
                                               \begin{itemize} \itemsep -2pt
                                               \item
                                                 Awarded Academic Excellence Award (IITK) given to top 5\% students on the basis of academic performance.
                                               \item
                                                 Secured AIR-248 out of 0.15 million students in JEE-Advanced 2013.
                                               \item
                                                 Selected for the KVPY (Kishore Vaigyanik Protsahan Yojana) in 2013
                                               \item
                                                 Qualified Regional Mathematical Olympiad(Rajasthan Region).
                                               \item
                                                 Among top 1\% in National Standard Examination in Physics 2013.
                                               \end{itemize}
  \\ \\ 
  % \noindent

  \textrm{\textsc{Internships and Experiences}} &
                                                  \begin{itemize} 
                                                  \item 
                                                    Altisource Business Solutions - Technology Intern \hfill  May'15-Jul'15
                                                    \begin{itemize} \itemsep -2pt
                                                    \item Worked on the projects ``Actor Equivalence and Identity Stitching'' and ``Event Data Clustering''
                                                      for the company's consumer analytics product `Pointilist'
                                                    \item Designed an algorithm for solving the actor equivalence problem (Established a relation between
                                                      actors and streaming events) for single/ Multiple platforms
                                                    \item Quantified various real life consumer attributes and clustered them to arrive at meaningful
                                                      inferences using Various clustering algorithms. 
                                                    \end{itemize}
                                                  \item
                                                    Worked on front-end of Website for AlumniContact \hfill Apr'14-Jul'14  \newline 
                                                    Programme, IIT Kanpur.
                                                  \end{itemize}
  \\ \\
  \textrm{\textsc{Projects}} &
                               \begin{itemize}
                               \item
                                 Sentiment Analysis of Social Media:   \hfill Aug'14
                                 \begin{itemize} \itemsep -2pt
                                 \item Application developed during Web-Dev, Takneek'14 and secured \nth{1} position.
                                 \item Provided an interface to analyse the past and present social sentiment of brands and their products.
                                 \item  Identify `good' and `bad' features of the product and suggest weak points to improve it.
                                 \end{itemize}
                               \item
                                 Image Trianglization                   \hfill May'14-Jul'14
                                 \begin{itemize}  \itemsep -2pt
                                 \item Renders image using only triangles and gives the sense of image.
                                 \item Basic GUI using Tkinter and webcam support.
                                 \item Uses Delaunay Triangulation so that image is aesthetically pleasing.
                                 \item Uses OpenCV for Image Processing and Manipulation.
                                 \end{itemize}
                               \item
                                 News Map                               \hfill Feb'14-Apr'14
                                 \begin{itemize} \itemsep -2pt
                                 \item A webapp that plots news on a world map.
                                 \item Uses Various APIs for location resolution, news fetching and map plotting.
                                 \end{itemize}
                               \end{itemize}
  \\ \\

   \textrm{\textsc{Relevant Coursework}} &  
                                        \begin{tabular}{p{69mm} p{60mm}} 
                                          CS210 - Data Structures and Algorithms A*  & CS201 - Discrete Mathematics A  \\
                                          ESC101 - Fundamentals of Computing A  &  CS251 - Computing Laboratory A    \\
                                          MTH102 - Linear Algebra and DE B  &  CS203 - Abstract Algebra B  \\
                                          MTH101 - Analytical Calculus B  & CS220 - Computer Organization B \\
                                          CS202 - Introduction to Logic B  &  MSO201 - Probability and Statistics A \\
                                          CS340 - Theory of Computation(Ongoing)  & CS330 - Operating Systems(Ongoing) \\
                                          CS345 - Algorithms II(Ongoing)\\
                                        \end{tabular}
\\ \\
  \textrm{\textsc{Technical Skills}} & 
                                       {\sl Development Environment :} Gentoo - Xmonad,i3 \newline
                                       {\sl Languages :} C/C++,Haskell,Scala,Python,Bash,JS,Dart.\newline
                                       {\sl Markup and Styling :} Html,scss/css,Markdown.\newline
                                       {\sl Libraries and frameworks :} Angular,OpenCV,Scrapy \newline
                                       {\sl Software :} Emacs, Vim, IntelliJ Idea.\newline
                                       {\sl Operating Systems:} Linux, Unix.\newline
                                       {\sl Other :} git,\LaTeX,Android App Development,Data Manipulation. \newline
  \\ \\
  \textrm{\textsc{Positions Of Responsibility}} &
                                                  \begin{itemize} \itemsep -2pt
                                                  \item Seceretary of Programming Club, IIT Kanpur. \hfill 2014-2015
                                                    \begin{itemize}
                                                      \item Part of a team of 15 members made to manage the activities of the Programming Club, IITK
                                                      \item Mentored freshmen for various Programming club competitions.
                                                      \end{itemize}
                                                  \end{itemize}

  \\ \\
\end{longtable}



%%%%%%%%%%%%%%%%%%%%% TRANSCRIPT %%%%%%%%%%%%%%%
% \newpage
% {\huge{\textrm{\textsc{Transcript}}}}


% \subsection*{Semester 1}

% \begin{tabular}{m{13cm} c c} 
%    %   \hline
    %     {Course} & {Credits} & {Grade} \\  \\
    %    %     \hline
    %     CHEMISTRY LABORATORY &  3 & A  \\ 
    %    %     \hline
    %     INTRODUCTION TO LINGUISTICS & 11 & A  \\
    %    %     \hline
    %     FUNDAMENTALS OF COMPUTING & 14 & A   \\
    %    %     \hline
    %     MATHEMATICS I & 14 & A    \\
    %    %     \hline
    %     MORNING EXERCISE & 3 & S \\
    %    %     \hline
    %     PHYSICS-II & 11 & A \\
    %    %     \hline
    %   \end{tabular}


    %     \subsection*{Semester 2}



    %     \begin{tabular}{m{13cm} c c} 
    %     %       \hline
    %     {Course} & {Credits} & {Grade} \\  \\
    %     %     \hline
    %     GENERAL CHEMISTRY &  8 & B  \\ 
    %     %     \hline
    %     INTRODUCTION TO BIOLOGY & 6 & B  	\\
    %     %     \hline
    %     MATHEMATICS - II & 11 & B    \\
    %     %     \hline
    %     PHYSICS LABORATORY & 3 & A    \\
    %     %     \hline
    %     EVENING EXERCISE & 3 & S \\
    %     %     \hline
    %     ENGINEERING GRAPHICS & 9 & A$^\ast$  \\
    %     %     \hline
    %   \end{tabular}



    %     \subsection*{Semester 3}



    %     \begin{tabular}{m{13cm} c c} 
    %    %       \hline
    %     {Course} & {Credits} & {Grade} \\ \\  
    %    %     \hline
    %     COMMUNICATION SKILLS: COMPOSITION &  5 & S  \\ 
    %    %     \hline
    %     MATHEMATICS FOR COMPUTER-SCIENCE - I & 9 & A  	\\
    %    %     \hline
    %     DATA STRUCTURES AND ALGORITHMS & 12 & A$^\ast$    \\
    %    %     \hline
    %     INTRODUCTION TO ELECTRONICS & 14 & A    \\
    %    %     \hline
    %     THERMODYNAMICS & 11 & B \\
    %    %     \hline
    %     MANUFACTURING PROCESSES I & 6 & B \\
    %    %     \hline
    %   \end{tabular}






\end{document}